\begin{frame}<1>[fragile,label=corpus]


\frametitle{SynPaFlex Corpus}

 
 \begin{tikzpicture}[node distance=2.1cm,scale=0.5, every node/.style={scale=0.5}]

 \node[cloud, cloud puffs=10.7, cloud ignores aspect, minimum width=2cm, minimum height=1cm, draw] (cloud) {Internet};
 
  \node[database,label={below:\textbf{SynPaFlex Corpus}},database radius=0.6cm,database segment height=0.4cm] (dataset) [below of=cloud,node distance=5cm]  {};
  
  \draw[->] (cloud)--(dataset) ;
  
  \node[align=center,rotate=90] at (0.2,-2.5)  {\textbf{Librivox.org}};
  
 
  
  
  \node[draw,text width=2.5cm,text height=1.5cm,visible on=<2->](prepossessing)[node distance=3.5cm, right of=dataset]{};
    \node[align=center,visible on=<2->] at (prepossessing.center)  {\textbf{Data Selection}};
    
    
    
    %Second Slide
    \node[text width=10cm,visible on=<2->] at (7,-0.5) (speaker){\-- A large corpus, single female speaker};
    %%
    \node[text width=10cm,visible on=<2->] at (7,-1) (quality){\-- Good audio signal quality and homogeneous voice};
    \node[text width=10cm,visible on=<2->] at (7,-1.5) (selecttext){\-- Various discourse styles and literary genres};
    \node[text width=10cm,visible on=<2->] at (7,-2) (datasizet){\-- \textbf{87 hours} of read speech };
    
    
    
    
    
    


    
    
\draw[->,thick,visible on=<2->] (dataset)--(prepossessing);
    
    
    
  
  \node[draw,text width=2.5cm,text height=1.5cm,align=center,visible on=<2->](normalisation)[node distance=3.5cm, right of=prepossessing]{};
    \node[align=center,visible on=<2->,visible on=<2->,visible on=<2->] at (normalisation.center)  {\textbf{Normalisation}};

\draw[->,thick,visible on=<2->,visible on=<2->] (prepossessing)--(normalisation);
  
  
      %%
    \node[text width=10cm,visible on=<2->] at (7,-2.5) (nomalisation){\-- Uniform text format and harmonized audio loudness  };
    
  
  
  
  \node[draw=blue,visible on=<2->](annotator)[right of=normalisation,node distance=3.5cm,text width=2.5cm,text height=1.5cm]{};
  
   \node[align=center,visible on=<2->] at (annotator.center)  {\textbf{Annotation}};
  
  \draw[->,thick,visible on=<2->] (normalisation)--(annotator);
  
       \node[text width=10cm,visible on=<2->] at (7,-3) (annotation){\-- Single  annotator}; 


%%EMOTION

 \node[circle,thick,minimum width=2cm,draw,visible on=<3->](emotion)[above right = 0.5cm and 0.9cm of annotator]{Emotions};
 
 \draw[->,thick,visible on=<3->] (annotator) edge[in=180,out=00] (emotion);
  

 \node[fill=blue!20,visible on=<3->](category)[right of=emotion,node distance=3.0cm,align=center]{\textbf{basic emotions}};
 

 \draw[->,thick,visible on=<3->] (emotion) -- (category);
 
 
 \node[below left=50pt and 30pt of emotion,draw=red!70,text width=8cm,visible on=<3>](reproduce){\textbf{Can we reproduce the manual annotation?}};
 

%\node[,visible on=<4->](intensity)[above of=category,node distance=1cm,align=center]{\textbf{Intensity}};
% \draw [->,thick,visible on=<4->] (emotion) edge[out=-0,in=180] (intensity);
 
 
 
 
 %\node[,visible on=<5->](state)[below of=category,minimum width=2 cm,node distance=1cm,align=center]{\textbf{Profile}};
 
 %\draw [->,thick,visible on=<5->] (emotion) edge[out=-0,in=180] (state);


%% Events Annotation

 \node[circle,thick,minimum width=2cm,draw,visible on=<7->](events)[below=5pt of emotion ]{Events};
\draw [->,thick,visible on=<7->] (annotator) edge[out=-0,in=180] (events);

 
 \node[right=10pt of events,visible on=<7->](phonetic){whispered voice};
 
 \draw [->,thick,visible on=<7->] (events) edge[out=-0,in=180] (phonetic);
 

%% Intonation
\node[below=20pt of events,circle,thick,minimum width=2cm,draw,visible on=<6->](intonation){Intonation};
 
 \draw [->,thick,visible on=<6->] (annotator) edge[out=-0,in=180] (intonation);
 
\node[right =10pt of intonation,visible on=<6->,align=left](pattern)[]{Question};

\draw [->,thick,visible on=<6->] (intonation) edge[out=-0,in=180] (pattern);










%%





%% Character annotation
 \node[below=5pt of intonation,circle,thick,minimum width=2 cm,minimum height=1cm,draw,visible on=<8->](characters)[visible on=<4-> ]{Characters};
 \draw[->,thick,visible on=<8->] (annotator) edge[out=-0,in=180] (characters);
 
 %% Characters Items
 
 \node[right=10pt of characters,visible on=<8->](charId)[]{Identifier};
 \draw[->,visible on=<8->,thick] (characters) -- (charId);
 
 
 %\node[below =5pt  of id,text width=1.5cm,visible on=<8->](vocalId){Vocal Identity};
 







%MuFsa
\frametitle<9>{MUFASA Corpus}
  \node[database,database radius=0.7cm,database segment height=0.5cm,visible on=<9->] (mufasa) [below of=dataset,node distance=4cm,visible on=<9->]  {};
   \node[align=center,visible on=<9->] at (mufasa.south){\textbf{MUFASA Corpus}};

\draw[draw=black!60,thick,->,visible on=<9->]  (cloud.west) -| (-1.5,-6.5) -- (mufasa) ;

\node[align=center,rotate=90,visible on=<9->] at (-1.3,-3.5)  {\textbf{Litterature audio.com}};

\draw[draw=black!60,thick,->,visible on=<9->]  (dataset) -- (mufasa) ;


\node[text width=10cm,visible on=<9->] at (7,-7) (mufasacontain){\-- 20 speaker  (10 Males/10 Females) };
\node[text width=10cm,visible on=<9->] at (7,-7.5) (parallel){\-- Parallel data, same text read by different speakers };






\node[align=center,rotate=90,visible on=<9->] at (0.2,-7)  {\textbf{update}};

\draw[draw=black!60,thick,->,visible on=<9->]  (mufasa)  -- (prepossessing) ;




%% For Future work

\frametitle<10>{Speaker dependent annotation}




     \begin{pgfonlayer}{background}
        \path (emotion.west)+(-0.5,2.) node (g) {};
        \path (events.east |- emotion.south)+(0.5,-2.5) node (h) {};
        \path[fill=blue!20,rounded corners, draw=black!50, dashed,visible on=<10->]
            (g) rectangle (h);
    \end{pgfonlayer}

\node[above=10pt of emotion,visible on=<10>] (spkdep){\textbf{Speaker dependent annotation}};





\frametitle<11>{Speaker independent annotation}

 \begin{pgfonlayer}{background}
        \path (intonation.west)+(-0.5,2.) node (g) {};
        \path (characters.east |- intonation.south)+(0.5,-2.5) node (h) {};
        \path[fill=red!20,rounded corners, draw=black!50, dashed,visible on=<11->]
            (g) rectangle (h);
    \end{pgfonlayer}

\node[below=10pt of characters,visible on=<11>] (spkinddep){\textbf{Speaker independent annotation}};


%\node[draw,line width=3pt,fit=(characters),visible on=<11>](discourse){Discourse};


\end{tikzpicture}

    
\end{frame}

